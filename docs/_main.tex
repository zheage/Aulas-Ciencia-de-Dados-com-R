% Options for packages loaded elsewhere
\PassOptionsToPackage{unicode}{hyperref}
\PassOptionsToPackage{hyphens}{url}
%
\documentclass[
]{book}
\usepackage{amsmath,amssymb}
\usepackage{lmodern}
\usepackage{iftex}
\ifPDFTeX
  \usepackage[T1]{fontenc}
  \usepackage[utf8]{inputenc}
  \usepackage{textcomp} % provide euro and other symbols
\else % if luatex or xetex
  \usepackage{unicode-math}
  \defaultfontfeatures{Scale=MatchLowercase}
  \defaultfontfeatures[\rmfamily]{Ligatures=TeX,Scale=1}
\fi
% Use upquote if available, for straight quotes in verbatim environments
\IfFileExists{upquote.sty}{\usepackage{upquote}}{}
\IfFileExists{microtype.sty}{% use microtype if available
  \usepackage[]{microtype}
  \UseMicrotypeSet[protrusion]{basicmath} % disable protrusion for tt fonts
}{}
\makeatletter
\@ifundefined{KOMAClassName}{% if non-KOMA class
  \IfFileExists{parskip.sty}{%
    \usepackage{parskip}
  }{% else
    \setlength{\parindent}{0pt}
    \setlength{\parskip}{6pt plus 2pt minus 1pt}}
}{% if KOMA class
  \KOMAoptions{parskip=half}}
\makeatother
\usepackage{xcolor}
\usepackage{color}
\usepackage{fancyvrb}
\newcommand{\VerbBar}{|}
\newcommand{\VERB}{\Verb[commandchars=\\\{\}]}
\DefineVerbatimEnvironment{Highlighting}{Verbatim}{commandchars=\\\{\}}
% Add ',fontsize=\small' for more characters per line
\usepackage{framed}
\definecolor{shadecolor}{RGB}{248,248,248}
\newenvironment{Shaded}{\begin{snugshade}}{\end{snugshade}}
\newcommand{\AlertTok}[1]{\textcolor[rgb]{0.94,0.16,0.16}{#1}}
\newcommand{\AnnotationTok}[1]{\textcolor[rgb]{0.56,0.35,0.01}{\textbf{\textit{#1}}}}
\newcommand{\AttributeTok}[1]{\textcolor[rgb]{0.77,0.63,0.00}{#1}}
\newcommand{\BaseNTok}[1]{\textcolor[rgb]{0.00,0.00,0.81}{#1}}
\newcommand{\BuiltInTok}[1]{#1}
\newcommand{\CharTok}[1]{\textcolor[rgb]{0.31,0.60,0.02}{#1}}
\newcommand{\CommentTok}[1]{\textcolor[rgb]{0.56,0.35,0.01}{\textit{#1}}}
\newcommand{\CommentVarTok}[1]{\textcolor[rgb]{0.56,0.35,0.01}{\textbf{\textit{#1}}}}
\newcommand{\ConstantTok}[1]{\textcolor[rgb]{0.00,0.00,0.00}{#1}}
\newcommand{\ControlFlowTok}[1]{\textcolor[rgb]{0.13,0.29,0.53}{\textbf{#1}}}
\newcommand{\DataTypeTok}[1]{\textcolor[rgb]{0.13,0.29,0.53}{#1}}
\newcommand{\DecValTok}[1]{\textcolor[rgb]{0.00,0.00,0.81}{#1}}
\newcommand{\DocumentationTok}[1]{\textcolor[rgb]{0.56,0.35,0.01}{\textbf{\textit{#1}}}}
\newcommand{\ErrorTok}[1]{\textcolor[rgb]{0.64,0.00,0.00}{\textbf{#1}}}
\newcommand{\ExtensionTok}[1]{#1}
\newcommand{\FloatTok}[1]{\textcolor[rgb]{0.00,0.00,0.81}{#1}}
\newcommand{\FunctionTok}[1]{\textcolor[rgb]{0.00,0.00,0.00}{#1}}
\newcommand{\ImportTok}[1]{#1}
\newcommand{\InformationTok}[1]{\textcolor[rgb]{0.56,0.35,0.01}{\textbf{\textit{#1}}}}
\newcommand{\KeywordTok}[1]{\textcolor[rgb]{0.13,0.29,0.53}{\textbf{#1}}}
\newcommand{\NormalTok}[1]{#1}
\newcommand{\OperatorTok}[1]{\textcolor[rgb]{0.81,0.36,0.00}{\textbf{#1}}}
\newcommand{\OtherTok}[1]{\textcolor[rgb]{0.56,0.35,0.01}{#1}}
\newcommand{\PreprocessorTok}[1]{\textcolor[rgb]{0.56,0.35,0.01}{\textit{#1}}}
\newcommand{\RegionMarkerTok}[1]{#1}
\newcommand{\SpecialCharTok}[1]{\textcolor[rgb]{0.00,0.00,0.00}{#1}}
\newcommand{\SpecialStringTok}[1]{\textcolor[rgb]{0.31,0.60,0.02}{#1}}
\newcommand{\StringTok}[1]{\textcolor[rgb]{0.31,0.60,0.02}{#1}}
\newcommand{\VariableTok}[1]{\textcolor[rgb]{0.00,0.00,0.00}{#1}}
\newcommand{\VerbatimStringTok}[1]{\textcolor[rgb]{0.31,0.60,0.02}{#1}}
\newcommand{\WarningTok}[1]{\textcolor[rgb]{0.56,0.35,0.01}{\textbf{\textit{#1}}}}
\usepackage{longtable,booktabs,array}
\usepackage{calc} % for calculating minipage widths
% Correct order of tables after \paragraph or \subparagraph
\usepackage{etoolbox}
\makeatletter
\patchcmd\longtable{\par}{\if@noskipsec\mbox{}\fi\par}{}{}
\makeatother
% Allow footnotes in longtable head/foot
\IfFileExists{footnotehyper.sty}{\usepackage{footnotehyper}}{\usepackage{footnote}}
\makesavenoteenv{longtable}
\usepackage{graphicx}
\makeatletter
\def\maxwidth{\ifdim\Gin@nat@width>\linewidth\linewidth\else\Gin@nat@width\fi}
\def\maxheight{\ifdim\Gin@nat@height>\textheight\textheight\else\Gin@nat@height\fi}
\makeatother
% Scale images if necessary, so that they will not overflow the page
% margins by default, and it is still possible to overwrite the defaults
% using explicit options in \includegraphics[width, height, ...]{}
\setkeys{Gin}{width=\maxwidth,height=\maxheight,keepaspectratio}
% Set default figure placement to htbp
\makeatletter
\def\fps@figure{htbp}
\makeatother
\setlength{\emergencystretch}{3em} % prevent overfull lines
\providecommand{\tightlist}{%
  \setlength{\itemsep}{0pt}\setlength{\parskip}{0pt}}
\setcounter{secnumdepth}{5}
\usepackage{booktabs}
\ifLuaTeX
  \usepackage{selnolig}  % disable illegal ligatures
\fi
\usepackage[]{natbib}
\bibliographystyle{plainnat}
\IfFileExists{bookmark.sty}{\usepackage{bookmark}}{\usepackage{hyperref}}
\IfFileExists{xurl.sty}{\usepackage{xurl}}{} % add URL line breaks if available
\urlstyle{same} % disable monospaced font for URLs
\hypersetup{
  pdftitle={CIÊNCIA DE DADOS COM LINGUAGEM R},
  pdfauthor={Richard Guilherme dos Santos},
  hidelinks,
  pdfcreator={LaTeX via pandoc}}

\title{CIÊNCIA DE DADOS COM LINGUAGEM R}
\author{Richard Guilherme dos Santos}
\date{}

\begin{document}
\maketitle

{
\setcounter{tocdepth}{1}
\tableofcontents
}
\hypertarget{introduuxe7uxe3o}{%
\chapter{Introdução}\label{introduuxe7uxe3o}}

Este livro tem como objetivo servir como guia para as aulas do curso Ciência de Dados com R. Nele apresentaremos os conceitos de:

\begin{enumerate}
\def\labelenumi{\arabic{enumi}.}
\tightlist
\item
  \textbf{Estatística Básica:} Nesta parte do curso abordaremos conceitos de estatística como variáveis, tipos de distribuições discretas e contínuas, medidas descritivas e distribuição normal.
\item
  \textbf{Manipulação de dados no R:} Neste tópico serão abordados as principais formas de manipulação de dados utilizando a linguagem R, com ênfase nas bibliotecas dplyr e tidyr. Além disso, abordaremos a criação de gráficos pelo pacote ggplot2.
\item
  \textbf{Modelos de Regressão Linear:} Parte final do curso, onde o aluno aprenderá sobre diagrama de dispersão, coeficiente de correlação linear, regressão linear simples, múltipla e regressão logística, ganhando a capacidade de começar a criar modelos utilizando a linguagem R.
\end{enumerate}

\hypertarget{introduuxe7uxe3o-a-probabilidade}{%
\chapter{Introdução a Probabilidade}\label{introduuxe7uxe3o-a-probabilidade}}

\hypertarget{introduuxe7uxe3o-ao-r}{%
\chapter{Introdução ao R}\label{introduuxe7uxe3o-ao-r}}

Aqui introduziremos alguns comandos da linguagem \texttt{R}. A linguagem utiliza de funções para realizar operações que vão desde leitura e manipulação de dados a operações matemáticas.

Comecemos criando um vetor de números:

\begin{Shaded}
\begin{Highlighting}[]
\NormalTok{x }\OtherTok{\textless{}{-}} \FunctionTok{c}\NormalTok{(}\DecValTok{1}\NormalTok{,}\DecValTok{3}\NormalTok{,}\DecValTok{2}\NormalTok{,}\DecValTok{5}\NormalTok{)}
\CommentTok{\# x = c(1,3,2,5) \# Também podemos utilizar "=" para atribuir variáveis}
\NormalTok{x}
\end{Highlighting}
\end{Shaded}

\begin{verbatim}
## [1] 1 3 2 5
\end{verbatim}

O comando acima combina os números 1,3,2 e 5 em um vetor de números e os salva em um objeto denominado x. Escrevemos x para recebermos os atributos do vetor.

A partir disto podemos utilizar outras funções para calcularmos informações destes atributos, como o tamanho de um vetor:

\begin{Shaded}
\begin{Highlighting}[]
\FunctionTok{length}\NormalTok{(x)}
\end{Highlighting}
\end{Shaded}

\begin{verbatim}
## [1] 4
\end{verbatim}

ou sua média:

\begin{Shaded}
\begin{Highlighting}[]
\FunctionTok{mean}\NormalTok{(x)}
\end{Highlighting}
\end{Shaded}

\begin{verbatim}
## [1] 2.75
\end{verbatim}

Há outros tipos de objetos que podem ser criados quando trabalhamos com \texttt{R}. Os mais importantes para manipulação de dados são as matrizes:

\begin{Shaded}
\begin{Highlighting}[]
\NormalTok{mat }\OtherTok{=} \FunctionTok{matrix}\NormalTok{(}\AttributeTok{data =} \FunctionTok{c}\NormalTok{(}\DecValTok{1}\NormalTok{,}\DecValTok{2}\NormalTok{,}\DecValTok{3}\NormalTok{,}\DecValTok{4}\NormalTok{), }\AttributeTok{nrow =} \DecValTok{2}\NormalTok{, }\AttributeTok{ncol =} \DecValTok{2}\NormalTok{,}
           \AttributeTok{byrow =} \ConstantTok{TRUE}\NormalTok{)}
\NormalTok{mat}
\end{Highlighting}
\end{Shaded}

\begin{verbatim}
##      [,1] [,2]
## [1,]    1    2
## [2,]    3    4
\end{verbatim}

\begin{quote}
Funções aceitam os mais diversos tipos de argumentos, para termos uma ideia de quais utilizarmos e seus atributos devemos consultar na biblioteca do \texttt{R}:

\begin{Shaded}
\begin{Highlighting}[]
\FunctionTok{help}\NormalTok{(matrix)}
\end{Highlighting}
\end{Shaded}
\end{quote}

E os data.frames, tabelas que aceitam dados de diversos tipos:

\begin{Shaded}
\begin{Highlighting}[]
\NormalTok{nomes }\OtherTok{=} \FunctionTok{c}\NormalTok{(}\StringTok{\textquotesingle{}Carol\textquotesingle{}}\NormalTok{, }\StringTok{\textquotesingle{}Alfredo\textquotesingle{}}\NormalTok{, }\StringTok{\textquotesingle{}Godoberto\textquotesingle{}}\NormalTok{)}
\NormalTok{idade }\OtherTok{=} \FunctionTok{c}\NormalTok{(}\DecValTok{18}\NormalTok{, }\DecValTok{23}\NormalTok{, }\DecValTok{19}\NormalTok{)}
\NormalTok{peso }\OtherTok{=} \FunctionTok{c}\NormalTok{(}\DecValTok{69}\NormalTok{, }\DecValTok{75}\NormalTok{, }\DecValTok{80}\NormalTok{)}
\NormalTok{altura }\OtherTok{=} \FunctionTok{c}\NormalTok{(}\FloatTok{1.70}\NormalTok{, }\FloatTok{1.80}\NormalTok{, }\FloatTok{1.85}\NormalTok{)}
\NormalTok{ICM }\OtherTok{=}\NormalTok{ peso}\SpecialCharTok{/}\NormalTok{altura}\SpecialCharTok{\^{}}\DecValTok{2}
\NormalTok{df }\OtherTok{=} \FunctionTok{data.frame}\NormalTok{(nomes, idade, peso, altura, ICM)}
\NormalTok{df}
\end{Highlighting}
\end{Shaded}

\begin{verbatim}
##       nomes idade peso altura      ICM
## 1     Carol    18   69   1.70 23.87543
## 2   Alfredo    23   75   1.80 23.14815
## 3 Godoberto    19   80   1.85 23.37473
\end{verbatim}

\hypertarget{medidas-descritivas}{%
\chapter{Medidas Descritivas}\label{medidas-descritivas}}

\hypertarget{tipos-de-variuxe1veis}{%
\section{Tipos de Variáveis}\label{tipos-de-variuxe1veis}}

Antes de analisarmos conjuntos de dados propriamente, é necessário termos um conhecimento sobre tipos de variáveis. Para isto, consideremos a seguinte tabela:

\begin{Shaded}
\begin{Highlighting}[]
\NormalTok{nome }\OtherTok{=} \FunctionTok{c}\NormalTok{(}\StringTok{\textquotesingle{}Guilherme\textquotesingle{}}\NormalTok{, }\StringTok{\textquotesingle{}Leon\textquotesingle{}}\NormalTok{, }\StringTok{\textquotesingle{}Nilce\textquotesingle{}}\NormalTok{)}
\NormalTok{est\_civil }\OtherTok{=} \FunctionTok{c}\NormalTok{(}\StringTok{\textquotesingle{}Solteiro\textquotesingle{}}\NormalTok{, }\StringTok{\textquotesingle{}Casado\textquotesingle{}}\NormalTok{, }\StringTok{\textquotesingle{}Casado\textquotesingle{}}\NormalTok{)}
\NormalTok{escolaridade }\OtherTok{=} \FunctionTok{c}\NormalTok{(}\StringTok{\textquotesingle{}Ensino médio completo\textquotesingle{}}\NormalTok{,}
                 \StringTok{\textquotesingle{}Pós{-}graduação\textquotesingle{}}\NormalTok{,}
                 \StringTok{\textquotesingle{}Superior completo\textquotesingle{}}\NormalTok{)}
\NormalTok{n\_filhos }\OtherTok{=} \FunctionTok{c}\NormalTok{(}\DecValTok{1}\NormalTok{, }\DecValTok{0}\NormalTok{, }\DecValTok{0}\NormalTok{)}
\NormalTok{salario }\OtherTok{=} \FunctionTok{c}\NormalTok{(}\DecValTok{1500}\NormalTok{, }\DecValTok{3000}\NormalTok{, }\DecValTok{3000}\NormalTok{)}
\NormalTok{idade }\OtherTok{=} \FunctionTok{c}\NormalTok{(}\DecValTok{21}\NormalTok{, }\DecValTok{39}\NormalTok{, }\DecValTok{32}\NormalTok{)}
\NormalTok{df }\OtherTok{=} \FunctionTok{data.frame}\NormalTok{(nome, est\_civil, escolaridade, n\_filhos, salario, idade)}
\FunctionTok{kable}\NormalTok{(df, }\AttributeTok{align =} \StringTok{\textquotesingle{}c\textquotesingle{}}\NormalTok{) }\CommentTok{\# Melhor visualização dos dados para este PDF}
\end{Highlighting}
\end{Shaded}

\begin{tabular}{c|c|c|c|c|c}
\hline
nome & est\_civil & escolaridade & n\_filhos & salario & idade\\
\hline
Guilherme & Solteiro & Ensino médio completo & 1 & 1500 & 21\\
\hline
Leon & Casado & Pós-graduação & 0 & 3000 & 39\\
\hline
Nilce & Casado & Superior completo & 0 & 3000 & 32\\
\hline
\end{tabular}

Variáveis como sexo, escolaridade e estado civil apresentam realizações de uma qualidade ou atributo do indivíduo pesquisado, enquanto outras como número de filhos, salário e idade apresentam números como resultados de uma contagem ou mensuração. Chamamos as do primeiro tipo de \textbf{qualitativas} e as do segundo de \textbf{quantitativas}

Cada uma das duas ainda pode ser dividida em dois tipos:

\begin{itemize}
\item
  \textbf{Variável qualitativa nominal:} atributos não apresentam uma ordem lógica;
\item
  \textbf{Variável qualitativa ordinal:} atributos apresentam uma ordem lógica bem estabelecida;
\item
  \textbf{Variável quantitativa discreta:} dados de contagem, assumem apenas valores inteiros;
\item
  \textbf{Variável quantitativa contínua:} dados que podem assumir qualquer tipo de valor.

  \includegraphics{figs/tipos_de_variaveis.png}
\end{itemize}

Muitas vezes queremos resumir estes dados, apresentando um ou mais valores que sejam representativos da série toda. Neste contexto entram às \textbf{medidas de posição e dispersão}.

\hypertarget{medidas-de-posiuxe7uxe3o}{%
\section{Medidas de Posição}\label{medidas-de-posiuxe7uxe3o}}

Usualmente utilizamos uma das seguintes medidas de posição (ou localização): \textbf{média, mediana ou moda}. Vamos as suas definições:

\begin{itemize}
\item
  \textbf{Moda:} valor mais frequente do conjunto de valores observados.
\item
  \textbf{Mediana:} valor que ocupa a posição central das observações quando estas estão ordenadas em ordem crescente.

  \begin{itemize}
  \tightlist
  \item
    Quando o número de observações for par, usa-se como mediana a média aritmética das duas observações centrais.
  \end{itemize}
\item
  \textbf{Média:} soma de todos os elementos do conjunto dividida pela quantidade de elementos do conjunto

  \[
  \overline{x} = \frac{x_1+x_2 + \dots + x_n}{n}
  \]
\end{itemize}

\hypertarget{medidas-de-dispersuxe3o}{%
\section{Medidas de Dispersão}\label{medidas-de-dispersuxe3o}}

O resumo de um conjunto de dados por uma única medida representativa de posição esconde toda a informação sobre a variabilidade de um conjunto de observações. Consideremos que cinco alunos realizaram cinco provas, obtendo as seguintes notas:

\begin{Shaded}
\begin{Highlighting}[]
\NormalTok{nomes }\OtherTok{=} \FunctionTok{c}\NormalTok{(}\StringTok{\textquotesingle{}alunoA\textquotesingle{}}\NormalTok{, }\StringTok{\textquotesingle{}alunoB\textquotesingle{}}\NormalTok{, }\StringTok{\textquotesingle{}alunoC\textquotesingle{}}\NormalTok{,}
          \StringTok{\textquotesingle{}alunoD\textquotesingle{}}\NormalTok{, }\StringTok{\textquotesingle{}alunoE\textquotesingle{}}\NormalTok{)}
\NormalTok{alunoA }\OtherTok{=} \FunctionTok{c}\NormalTok{(}\DecValTok{3}\NormalTok{,}\DecValTok{4}\NormalTok{,}\DecValTok{5}\NormalTok{,}\DecValTok{6}\NormalTok{,}\DecValTok{7}\NormalTok{)}
\NormalTok{alunoB }\OtherTok{=} \FunctionTok{c}\NormalTok{(}\DecValTok{1}\NormalTok{,}\DecValTok{3}\NormalTok{,}\DecValTok{5}\NormalTok{,}\DecValTok{7}\NormalTok{,}\DecValTok{9}\NormalTok{)}
\NormalTok{alunoC }\OtherTok{=} \FunctionTok{c}\NormalTok{(}\DecValTok{5}\NormalTok{,}\DecValTok{5}\NormalTok{,}\DecValTok{5}\NormalTok{,}\DecValTok{5}\NormalTok{,}\DecValTok{5}\NormalTok{)}
\NormalTok{alunoD }\OtherTok{=} \FunctionTok{c}\NormalTok{(}\DecValTok{3}\NormalTok{,}\DecValTok{5}\NormalTok{,}\DecValTok{5}\NormalTok{,}\DecValTok{5}\NormalTok{,}\DecValTok{7}\NormalTok{)}
\NormalTok{alunoE }\OtherTok{=} \FunctionTok{c}\NormalTok{(}\DecValTok{3}\NormalTok{,}\DecValTok{5}\NormalTok{,}\DecValTok{5}\NormalTok{,}\DecValTok{6}\NormalTok{,}\DecValTok{6}\NormalTok{)}
\NormalTok{df }\OtherTok{=} \FunctionTok{data.frame}\NormalTok{(alunoA, alunoB, alunoC, alunoD, alunoE)}
\FunctionTok{row.names}\NormalTok{(df) }\OtherTok{=}\NormalTok{ nomes}
\NormalTok{df}
\end{Highlighting}
\end{Shaded}

\begin{verbatim}
##        alunoA alunoB alunoC alunoD alunoE
## alunoA      3      1      5      3      3
## alunoB      4      3      5      5      5
## alunoC      5      5      5      5      5
## alunoD      6      7      5      5      6
## alunoE      7      9      5      7      6
\end{verbatim}

\hypertarget{quantis-empuxedricos}{%
\section{Quantis Empíricos}\label{quantis-empuxedricos}}

\hypertarget{box-plot}{%
\section{Box Plot}\label{box-plot}}

\hypertarget{transformauxe7uxf5es}{%
\section{Transformações}\label{transformauxe7uxf5es}}

\[ y=x^2\]

\hypertarget{tipos-de-distribuiuxe7uxf5es-discretas}{%
\chapter{Tipos de Distribuições Discretas}\label{tipos-de-distribuiuxe7uxf5es-discretas}}

\hypertarget{tipos-de-distribuiuxe7uxf5es-contuxednuas}{%
\chapter{Tipos de Distribuições Contínuas}\label{tipos-de-distribuiuxe7uxf5es-contuxednuas}}

\hypertarget{introduuxe7uxe3o-as-bibliotecas-do-r}{%
\chapter{Introdução as bibliotecas do R}\label{introduuxe7uxe3o-as-bibliotecas-do-r}}

\hypertarget{dplyr}{%
\section{Dplyr}\label{dplyr}}

\hypertarget{tidyr}{%
\section{Tidyr}\label{tidyr}}

\hypertarget{ggplot2}{%
\section{GGPlot2}\label{ggplot2}}

\hypertarget{regressuxe3o-linear}{%
\chapter{Regressão Linear}\label{regressuxe3o-linear}}

  \bibliography{book.bib,packages.bib}

\end{document}

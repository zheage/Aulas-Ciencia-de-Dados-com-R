% Options for packages loaded elsewhere
\PassOptionsToPackage{unicode}{hyperref}
\PassOptionsToPackage{hyphens}{url}
%
\documentclass[
]{book}
\usepackage{amsmath,amssymb}
\usepackage{lmodern}
\usepackage{iftex}
\ifPDFTeX
  \usepackage[T1]{fontenc}
  \usepackage[utf8]{inputenc}
  \usepackage{textcomp} % provide euro and other symbols
\else % if luatex or xetex
  \usepackage{unicode-math}
  \defaultfontfeatures{Scale=MatchLowercase}
  \defaultfontfeatures[\rmfamily]{Ligatures=TeX,Scale=1}
\fi
% Use upquote if available, for straight quotes in verbatim environments
\IfFileExists{upquote.sty}{\usepackage{upquote}}{}
\IfFileExists{microtype.sty}{% use microtype if available
  \usepackage[]{microtype}
  \UseMicrotypeSet[protrusion]{basicmath} % disable protrusion for tt fonts
}{}
\makeatletter
\@ifundefined{KOMAClassName}{% if non-KOMA class
  \IfFileExists{parskip.sty}{%
    \usepackage{parskip}
  }{% else
    \setlength{\parindent}{0pt}
    \setlength{\parskip}{6pt plus 2pt minus 1pt}}
}{% if KOMA class
  \KOMAoptions{parskip=half}}
\makeatother
\usepackage{xcolor}
\usepackage{longtable,booktabs,array}
\usepackage{calc} % for calculating minipage widths
% Correct order of tables after \paragraph or \subparagraph
\usepackage{etoolbox}
\makeatletter
\patchcmd\longtable{\par}{\if@noskipsec\mbox{}\fi\par}{}{}
\makeatother
% Allow footnotes in longtable head/foot
\IfFileExists{footnotehyper.sty}{\usepackage{footnotehyper}}{\usepackage{footnote}}
\makesavenoteenv{longtable}
\usepackage{graphicx}
\makeatletter
\def\maxwidth{\ifdim\Gin@nat@width>\linewidth\linewidth\else\Gin@nat@width\fi}
\def\maxheight{\ifdim\Gin@nat@height>\textheight\textheight\else\Gin@nat@height\fi}
\makeatother
% Scale images if necessary, so that they will not overflow the page
% margins by default, and it is still possible to overwrite the defaults
% using explicit options in \includegraphics[width, height, ...]{}
\setkeys{Gin}{width=\maxwidth,height=\maxheight,keepaspectratio}
% Set default figure placement to htbp
\makeatletter
\def\fps@figure{htbp}
\makeatother
\setlength{\emergencystretch}{3em} % prevent overfull lines
\providecommand{\tightlist}{%
  \setlength{\itemsep}{0pt}\setlength{\parskip}{0pt}}
\setcounter{secnumdepth}{5}
\usepackage{booktabs}
\ifLuaTeX
  \usepackage{selnolig}  % disable illegal ligatures
\fi
\usepackage[]{natbib}
\bibliographystyle{plainnat}
\IfFileExists{bookmark.sty}{\usepackage{bookmark}}{\usepackage{hyperref}}
\IfFileExists{xurl.sty}{\usepackage{xurl}}{} % add URL line breaks if available
\urlstyle{same} % disable monospaced font for URLs
\hypersetup{
  pdftitle={CIÊNCIA DE DADOS COM LINGUAGEM R},
  pdfauthor={Richard Guilherme dos Santos},
  hidelinks,
  pdfcreator={LaTeX via pandoc}}

\title{CIÊNCIA DE DADOS COM LINGUAGEM R}
\author{Richard Guilherme dos Santos}
\date{}

\begin{document}
\maketitle

{
\setcounter{tocdepth}{1}
\tableofcontents
}
\hypertarget{introduuxe7uxe3o}{%
\chapter{Introdução}\label{introduuxe7uxe3o}}

Este livro tem como objetivo servir como guia para as aulas do curso Ciência de Dados com R. Nele apresentaremos os conceitos de:

\begin{enumerate}
\def\labelenumi{\arabic{enumi}.}
\tightlist
\item
  \textbf{Estatística Básica:} Nesta parte do curso abordaremos conceitos de estatística como variáveis, tipos de distribuições discretas e contínuas, medidas descritivas e distribuição normal.
\item
  \textbf{Manipulação de dados no R:} Neste tópico serão abordados as principais formas de manipulação de dados utilizando a linguagem R, com ênfase nas bibliotecas dplyr e tidyr. Além disso, abordaremos a criação de gráficos pelo pacote ggplot2.
\item
  \textbf{Modelos de Regressão Linear:} Parte final do curso, onde o aluno aprenderá sobre diagrama de dispersão, coeficiente de correlação linear, regressão linear simples, múltipla e regressão logística, ganhando a capacidade de começar a criar modelos utilizando a linguagem R.
\end{enumerate}

\hypertarget{introduuxe7uxe3o-a-probabilidade}{%
\chapter{Introdução a Probabilidade}\label{introduuxe7uxe3o-a-probabilidade}}

\hypertarget{introduuxe7uxe3o-ao-r}{%
\chapter{Introdução ao R}\label{introduuxe7uxe3o-ao-r}}

\hypertarget{comandos-buxe1sicos}{%
\section{Comandos Básicos}\label{comandos-buxe1sicos}}

\hypertarget{medidas-descritivas}{%
\chapter{Medidas Descritivas}\label{medidas-descritivas}}

\[ y=x^2\]

\hypertarget{tipos-de-distribuiuxe7uxf5es-discretas}{%
\chapter{Tipos de Distribuições Discretas}\label{tipos-de-distribuiuxe7uxf5es-discretas}}

\hypertarget{tipos-de-distribuiuxe7uxf5es-contuxednuas}{%
\chapter{Tipos de Distribuições Contínuas}\label{tipos-de-distribuiuxe7uxf5es-contuxednuas}}

\hypertarget{introduuxe7uxe3o-as-bibliotecas-do-r}{%
\chapter{Introdução as bibliotecas do R}\label{introduuxe7uxe3o-as-bibliotecas-do-r}}

\hypertarget{dplyr}{%
\section{Dplyr}\label{dplyr}}

\hypertarget{tidyr}{%
\section{Tidyr}\label{tidyr}}

\hypertarget{ggplot2}{%
\section{GGPlot2}\label{ggplot2}}

\hypertarget{regressuxe3o-linear}{%
\chapter{Regressão Linear}\label{regressuxe3o-linear}}

  \bibliography{book.bib,packages.bib}

\end{document}
